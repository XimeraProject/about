\documentclass{ximera}
\title{Security}

\begin{document}
\begin{abstract}
  The Ximera development team is committed to protecting your data.
\end{abstract}
\maketitle

The Ximera team encourages users to \link[join the Ximera mailing
list]{http://go.osu.edu/ximera-mailing-list} in order to receive
updates on new features, disclosures about bugs, and discussion of
security issues around Ximera and its data.

\section{Threat modeling analysis}

For Ximera, security is not an afterthought but has been baked into
the design from the very beginning.  This means security is expressed
not only through code, but also through the team's workflow and
through the underlying design.  For instance, because the long-term
goal of Ximera is a federated network of lightweight servers (and
ultimately \textit{no servers at all} through IPFS), the highest
levels of permission are granted to users via GPG keys which already
play well with a federated web of trust.

\subsection{Security Objectives}

Security begins by specifying our objectives.  In terms of user
stories, there are two main populations who both need controlled
access to data to fulfill their own objectives.
\begin{itemize}
\item As an instructor, I want to see how a particular student is
  doing on assignments related to an upcoming lecture.
\item As an author, I want to understand how students, in the
  aggregate, are engaging with my content.
\end{itemize}
The role of ``instructor'' is defined via LTI, meaning that a
student-instructor relationship exists between $A$ and $B$ when user
$A$ with a student role and user $B$ with an instructor role both
share the same course context.  That instructor, then, is permitted to
see the name and work of his/her student in real-time.

The second population---that of authors---are provided not access to
specific students, but access to de-identified event logs for their
students (meaning users engaging with content in their repository).
Authors are \textit{not} identified through LTI, but rather through
proof-of-possession of a private key.  This facilitates a loosely
federated model, where the same author may have content on multiple
Ximera servers---without permitting those who control the other
servers to spoof the identity of an author.  GPG helps to provide a
consistent digital identity for authors across multiple sites.

To ensure that the team understands how these objectives are met,
Ximera includes robust logging.

To prepare for the envisioned decentralized future, Ximera, insofar as
is possible, pushes answer validation and the like to the learner's
machine.  This addresses a performance issue by avoiding round-trips
for answer validation, but it also reduces our threat surface because
the server is \textit{not} executing code provided it by the student.
When there is a pedagogical reason to rely on a computer algebra
system for answer adjudication, the Ximera team relies on SageMath and
the well-sandboxed SageCell servers.

\subsection{Threat vectors and their mitigation}

That authors are permitted to publish arbitrary HTML is the most
significant threat vector.  This could happen if it were possible to
spoof the identity of an author who is permitted to publish a
classroom activity.  At Ximera Workshops, authors are trained to
protect their GPG keys with a strong passphrase entered via a pinentry
program.

The second main threat is a denial of service attack; because Ximera
is committed to open education, it will always be possible for random
internet users to explore the educational content created on the
platform, and there is some cost (e.g., a few milliseconds) involved
in serving each learner.  Plans to add additional VMs are expected to
reduce this threat.

To protect against information disclosure, the \texttt{xake} tool only
exports de-identified data.  There is no endpoint for exporting data
from mongodb.  Similarly, there are no ``superuser'' accounts

To address elevation of privilege, there is a ``superuser'' account
but its ability is to set flags---but the only power of these flags is
to decorate a user's account with information and the flags themselves
do not provide additional permissions.

Finally, there are threats around students cheating on assignments.
This is made worse by the fact that Ximera, being committed to
accessibility and open-source, expects authors to make source
available, and that source code, in some fashion, includes ``the right
answers'' which could be replayed.  Ultimately the Ximera team
recommends the use of Ximera for low-stakes formative assessment.
Additional tooling would be needed for high-stakes assessment.

\section{How we protect your data}

The Ximera team is supported with DevOps provided by Ohio State's
College of Arts and Sciences Technology Team (ASCTech).  That team
manages the firewalled virtual machines and ensures that they are
up-to-date.  We leverage existing security frameworks, like automated
threat detection tools for cross-site scripting and injection attacks.
We also leverage existing tools for code auditing, like GitHub's
automatic threat detection for verifying that the NPM packages that
Ximera depends on are free from known vulnerabilities, thereby
ensuring that we aren't using components with known vulnerabilities.
Involving ASCTech also helps protect the team from potential security
holes due to misconfiguration of the servers and makes recommendations
to ensure secure access to the database.

\end{document}

